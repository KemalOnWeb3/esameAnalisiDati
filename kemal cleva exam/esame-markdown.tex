% Options for packages loaded elsewhere
\PassOptionsToPackage{unicode}{hyperref}
\PassOptionsToPackage{hyphens}{url}
%
\documentclass[
  ignorenonframetext,
]{beamer}
\usepackage{pgfpages}
\setbeamertemplate{caption}[numbered]
\setbeamertemplate{caption label separator}{: }
\setbeamercolor{caption name}{fg=normal text.fg}
\beamertemplatenavigationsymbolsempty
% Prevent slide breaks in the middle of a paragraph
\widowpenalties 1 10000
\raggedbottom
\setbeamertemplate{part page}{
  \centering
  \begin{beamercolorbox}[sep=16pt,center]{part title}
    \usebeamerfont{part title}\insertpart\par
  \end{beamercolorbox}
}
\setbeamertemplate{section page}{
  \centering
  \begin{beamercolorbox}[sep=12pt,center]{part title}
    \usebeamerfont{section title}\insertsection\par
  \end{beamercolorbox}
}
\setbeamertemplate{subsection page}{
  \centering
  \begin{beamercolorbox}[sep=8pt,center]{part title}
    \usebeamerfont{subsection title}\insertsubsection\par
  \end{beamercolorbox}
}
\AtBeginPart{
  \frame{\partpage}
}
\AtBeginSection{
  \ifbibliography
  \else
    \frame{\sectionpage}
  \fi
}
\AtBeginSubsection{
  \frame{\subsectionpage}
}
\usepackage{amsmath,amssymb}
\usepackage{lmodern}
\usepackage{iftex}
\ifPDFTeX
  \usepackage[T1]{fontenc}
  \usepackage[utf8]{inputenc}
  \usepackage{textcomp} % provide euro and other symbols
\else % if luatex or xetex
  \usepackage{unicode-math}
  \defaultfontfeatures{Scale=MatchLowercase}
  \defaultfontfeatures[\rmfamily]{Ligatures=TeX,Scale=1}
\fi
% Use upquote if available, for straight quotes in verbatim environments
\IfFileExists{upquote.sty}{\usepackage{upquote}}{}
\IfFileExists{microtype.sty}{% use microtype if available
  \usepackage[]{microtype}
  \UseMicrotypeSet[protrusion]{basicmath} % disable protrusion for tt fonts
}{}
\makeatletter
\@ifundefined{KOMAClassName}{% if non-KOMA class
  \IfFileExists{parskip.sty}{%
    \usepackage{parskip}
  }{% else
    \setlength{\parindent}{0pt}
    \setlength{\parskip}{6pt plus 2pt minus 1pt}}
}{% if KOMA class
  \KOMAoptions{parskip=half}}
\makeatother
\usepackage{xcolor}
\newif\ifbibliography
\usepackage{color}
\usepackage{fancyvrb}
\newcommand{\VerbBar}{|}
\newcommand{\VERB}{\Verb[commandchars=\\\{\}]}
\DefineVerbatimEnvironment{Highlighting}{Verbatim}{commandchars=\\\{\}}
% Add ',fontsize=\small' for more characters per line
\usepackage{framed}
\definecolor{shadecolor}{RGB}{248,248,248}
\newenvironment{Shaded}{\begin{snugshade}}{\end{snugshade}}
\newcommand{\AlertTok}[1]{\textcolor[rgb]{0.94,0.16,0.16}{#1}}
\newcommand{\AnnotationTok}[1]{\textcolor[rgb]{0.56,0.35,0.01}{\textbf{\textit{#1}}}}
\newcommand{\AttributeTok}[1]{\textcolor[rgb]{0.77,0.63,0.00}{#1}}
\newcommand{\BaseNTok}[1]{\textcolor[rgb]{0.00,0.00,0.81}{#1}}
\newcommand{\BuiltInTok}[1]{#1}
\newcommand{\CharTok}[1]{\textcolor[rgb]{0.31,0.60,0.02}{#1}}
\newcommand{\CommentTok}[1]{\textcolor[rgb]{0.56,0.35,0.01}{\textit{#1}}}
\newcommand{\CommentVarTok}[1]{\textcolor[rgb]{0.56,0.35,0.01}{\textbf{\textit{#1}}}}
\newcommand{\ConstantTok}[1]{\textcolor[rgb]{0.00,0.00,0.00}{#1}}
\newcommand{\ControlFlowTok}[1]{\textcolor[rgb]{0.13,0.29,0.53}{\textbf{#1}}}
\newcommand{\DataTypeTok}[1]{\textcolor[rgb]{0.13,0.29,0.53}{#1}}
\newcommand{\DecValTok}[1]{\textcolor[rgb]{0.00,0.00,0.81}{#1}}
\newcommand{\DocumentationTok}[1]{\textcolor[rgb]{0.56,0.35,0.01}{\textbf{\textit{#1}}}}
\newcommand{\ErrorTok}[1]{\textcolor[rgb]{0.64,0.00,0.00}{\textbf{#1}}}
\newcommand{\ExtensionTok}[1]{#1}
\newcommand{\FloatTok}[1]{\textcolor[rgb]{0.00,0.00,0.81}{#1}}
\newcommand{\FunctionTok}[1]{\textcolor[rgb]{0.00,0.00,0.00}{#1}}
\newcommand{\ImportTok}[1]{#1}
\newcommand{\InformationTok}[1]{\textcolor[rgb]{0.56,0.35,0.01}{\textbf{\textit{#1}}}}
\newcommand{\KeywordTok}[1]{\textcolor[rgb]{0.13,0.29,0.53}{\textbf{#1}}}
\newcommand{\NormalTok}[1]{#1}
\newcommand{\OperatorTok}[1]{\textcolor[rgb]{0.81,0.36,0.00}{\textbf{#1}}}
\newcommand{\OtherTok}[1]{\textcolor[rgb]{0.56,0.35,0.01}{#1}}
\newcommand{\PreprocessorTok}[1]{\textcolor[rgb]{0.56,0.35,0.01}{\textit{#1}}}
\newcommand{\RegionMarkerTok}[1]{#1}
\newcommand{\SpecialCharTok}[1]{\textcolor[rgb]{0.00,0.00,0.00}{#1}}
\newcommand{\SpecialStringTok}[1]{\textcolor[rgb]{0.31,0.60,0.02}{#1}}
\newcommand{\StringTok}[1]{\textcolor[rgb]{0.31,0.60,0.02}{#1}}
\newcommand{\VariableTok}[1]{\textcolor[rgb]{0.00,0.00,0.00}{#1}}
\newcommand{\VerbatimStringTok}[1]{\textcolor[rgb]{0.31,0.60,0.02}{#1}}
\newcommand{\WarningTok}[1]{\textcolor[rgb]{0.56,0.35,0.01}{\textbf{\textit{#1}}}}
\setlength{\emergencystretch}{3em} % prevent overfull lines
\providecommand{\tightlist}{%
  \setlength{\itemsep}{0pt}\setlength{\parskip}{0pt}}
\setcounter{secnumdepth}{-\maxdimen} % remove section numbering
\ifLuaTeX
  \usepackage{selnolig}  % disable illegal ligatures
\fi
\IfFileExists{bookmark.sty}{\usepackage{bookmark}}{\usepackage{hyperref}}
\IfFileExists{xurl.sty}{\usepackage{xurl}}{} % add URL line breaks if available
\urlstyle{same} % disable monospaced font for URLs
\hypersetup{
  pdftitle={Inequality of incomes},
  pdfauthor={Kemal Cleva 160593},
  hidelinks,
  pdfcreator={LaTeX via pandoc}}

\title{Inequality of incomes}
\author{Kemal Cleva 160593}
\date{}

\begin{document}
\frame{\titlepage}

\begin{frame}{Disuguaglianza dei redditi}
\protect\hypertarget{disuguaglianza-dei-redditi}{}
\begin{quote}
Parlando di disuguaglianza dei redditi si pensa subito che con l'andare
degli anni essa aumenti irreversibilitmente in tutto il mondo.
Utilizzeremo la scienza dei dati per scoprire come vanno effettivamente
le cose.
\end{quote}

\begin{quote}
Il seguente plot espone in maniera generale l'andamento negli anni (dal
1900 al 2016) della disugualianza dei redditi.
\end{quote}
\end{frame}

\begin{frame}{Plot delle nazioni}
\protect\hypertarget{plot-delle-nazioni}{}
\end{frame}

\begin{frame}{Gini}
\protect\hypertarget{gini}{}
\begin{quote}
Il rapporto Gini, è una misura della dispersione statistica destinata a
rappresentare la disuguaglianza di reddito o la disuguaglianza di
ricchezza all'interno di una nazione.
\end{quote}

\begin{quote}
Di seguito vedremo un boxplot che prende come variabili Gini e Paese.
\end{quote}
\end{frame}

\begin{frame}{BoxPlot Gini su nazioni}
\protect\hypertarget{boxplot-gini-su-nazioni}{}
\end{frame}

\begin{frame}{Modello e predizioni}
\protect\hypertarget{modello-e-predizioni}{}
\begin{quote}
Con una predizione possiamo vedere chiaramente qual è la tendenza
(direzione) di tutti i paesi in termini di disuguaglianza di reddito.
Con la funzione geom\_line() si può creare una linea di tendenza
(predizione).
\end{quote}
\end{frame}

\begin{frame}{GGPlot dei valori osservati e predetti}
\protect\hypertarget{ggplot-dei-valori-osservati-e-predetti}{}
\end{frame}

\begin{frame}{Import}
\protect\hypertarget{import}{}
\begin{itemize}[<+->]
\tightlist
\item
  first you must import your data into R
\item
  this typically means that you take data stored in a file, database, or
  web API, and load it into a data frame in R
\item
  if you can’t get your data into R, you can’t do data science on it
\end{itemize}
\end{frame}

\begin{frame}{Tidy}
\protect\hypertarget{tidy}{}
\begin{itemize}[<+->]
\tightlist
\item
  once you’ve imported your data, it is a good idea to \textbf{tidy}
  it
\item
  tidying your data means storing it in a consistent form that matches
  the semantics of the dataset with the way it is stored
\item
  in brief, when your data is tidy, each column is a variable and each
  row is an observation
\item
  tidy data is important because the consistent structure lets you focus
  on questions about the data, not fighting to get the data into the
  right form to answer your questions
\end{itemize}
\end{frame}

\begin{frame}{Transform}
\protect\hypertarget{transform}{}
\begin{itemize}[<+->]
\tightlist
\item
  once you have tidy data, a common first step is to \textbf{transform}
  (or \textbf{query}) it
\item
  transformation includes:

  \begin{itemize}[<+->]
  \tightlist
  \item
    narrowing in on observations of interest (like all people in one
    city, or all data from the last year)
  \item
    creating new variables that are functions of existing variables
    (like computing velocity from speed and time)
  \item
    calculating a set of summary statistics (like counts or means)
  \end{itemize}
\item
  together, tidying and transforming are called \textbf{wrangling},
  because getting your data in a form that's natural to work with often
  feels like a fight!
\end{itemize}
\end{frame}

\begin{frame}{Visualize and model}
\protect\hypertarget{visualize-and-model}{}
\begin{itemize}[<+->]
\tightlist
\item
  once you have tidy data with the variables you need, there are two
  main engines of knowledge generation: \textbf{visualisation} and
  \textbf{modelling}
\item
  these have complementary strengths and weaknesses so any real analysis
  will iterate between them many times
\end{itemize}
\end{frame}

\begin{frame}{Visualize}
\protect\hypertarget{visualize}{}
\begin{itemize}[<+->]
\tightlist
\item
  \textbf{visualisation} is a fundamentally human activity
\item
  good visualisation will show you things that you did not expect, or
  raise new questions about the data
\item
  a good visualisation might also hint that you're asking the wrong
  question, or you need to collect different data
\item
  visualisations can surprise you, but \textbf{don't scale} particularly
  well because they require a human to interpret them
\end{itemize}
\end{frame}

\begin{frame}
\end{frame}

\begin{frame}{Model}
\protect\hypertarget{model}{}
\begin{itemize}[<+->]
\tightlist
\item
  \textbf{models} are complementary tools to visualisation
\item
  the goal of a model is to provide a simple low-dimensional summary of
  a dataset
\item
  ideally, the model will capture true \textbf{signals} (i.e.~patterns
  generated by the phenomenon of interest) and ignore \textbf{noise}
  (i.e.~random variation that you’re not interested in)
\item
  models are a fundamentally mathematical or computational tool, so they
  generally scale well
\item
  but ``\emph{the map is not the territory}'': every model makes
  assumptions, and these make a difference between reality and a model
  of reality
\end{itemize}
\end{frame}

\begin{frame}{Communicate}
\protect\hypertarget{communicate}{}
\begin{itemize}[<+->]
\tightlist
\item
  the last step of data science is \textbf{communication}, an absolutely
  critical part of any data analysis project
\item
  it doesn't matter how well your models and visualisation have led you
  to understand the data unless you can also communicate your results to
  others, including the future you
\end{itemize}
\end{frame}

\begin{frame}{Hypothesis generation or confirmation?}
\protect\hypertarget{hypothesis-generation-or-confirmation}{}
\begin{itemize}[<+->]
\tightlist
\item
  it’s possible to divide data analysis into two camps: hypothesis
  generation and hypothesis confirmation
\item
  the focus of this course is on \textbf{hypothesis generation}, or data
  exploration
\item
  here you’ll look deeply at the data and, in combination with your
  subject knowledge, generate many interesting hypotheses to help
  explain why the data behaves the way it does
\item
  you evaluate the hypotheses informally, using your skepticism to
  challenge the data in multiple ways
\end{itemize}
\end{frame}

\begin{frame}[fragile]{Tidyverse}
\protect\hypertarget{tidyverse}{}
We'll follow the data science graph using the
\href{https://www.tidyverse.org}{tidyverse} approach developed by Hadley
Wickham

\begin{quote}
The tidyverse is an opinionated collection of R packages designed for
data science. All packages share an underlying design philosophy,
grammar, and data structures. Hadley Wickham
\end{quote}

Install the complete tidyverse with:

\begin{Shaded}
\begin{Highlighting}[]
\FunctionTok{install.packages}\NormalTok{(}\StringTok{"tidyverse"}\NormalTok{)}
\end{Highlighting}
\end{Shaded}
\end{frame}

\begin{frame}{Big data}
\protect\hypertarget{big-data}{}
\begin{quote}
Big data refers to datasets whose size is beyond the ability of typical
database software tools to capture, store, manage, and analyze
\end{quote}

\begin{itemize}[<+->]
\tightlist
\item
  this course proudly focuses on small, in-memory datasets
\item
  this is the right place to start because you can’t tackle big data
  unless you have experience with small data
\end{itemize}
\end{frame}

\begin{frame}{Big data solutions}
\protect\hypertarget{big-data-solutions}{}
\begin{enumerate}[<+->]
\tightlist
\item
  while the complete data might be big, often the data needed to answer
  a specific question is small; you might be able to find a
  \textbf{sample} or summary that fits in memory and still allows you to
  answer the question that you're interested in
\item
  you can \textbf{scale up} or \textbf{scale out} your hardware
\item
  you can store (in secondary memory) your dataset in a
  \textbf{database} and use packages like
  \href{https://cran.r-project.org/web/packages/dbplyr/index.html}{dbplyr}
  to work with remote database tables as if they are in-memory data
  frames
\item
  you can take advantage of a \textbf{cloud storage and computing
  system}, like
  \href{https://cloud.google.com/bigquery?hl=en}{BigQuery}, and access
  it from R with package
  \href{https://CRAN.R-project.org/package=bigrquery}{bigrquery}
\item
  finally, you can use a \textbf{cluster computing platform} that allows
  you to spread your data and your computations across multiple machines
  and work with packages like \href{https://spark.rstudio.com}{sparklyr}
\end{enumerate}
\end{frame}

\begin{frame}{Blockchain}
\protect\hypertarget{blockchain}{}
\begin{itemize}[<+->]
\tightlist
\item
  informally, a \textbf{blockchain} is a time-stamped record of any kind
  of information, organized into blocks that are chained together
\item
  more formally, a blockchain is:

  \begin{itemize}[<+->]
  \tightlist
  \item
    a \textbf{distributed system}
  \item
    using \textbf{cryptography}
  \item
    to secure an evolving \textbf{consensus}
  \item
    about a \textbf{token} with economic value
  \end{itemize}
\end{itemize}
\end{frame}

\end{document}
